\documentclass[a4paper,12pt]{article}
\usepackage[utf8]{vietnam}
\usepackage{amsmath, amssymb}
\usepackage{graphicx}
\usepackage{algorithm}
\usepackage[noend]{algpseudocode}
\usepackage{hyperref}
\usepackage[a4paper, margin=1in]{geometry}

\title{Phương pháp phân bổ nguồn lực cho nhiều dự án với thuật toán DLHS điều chỉnh}
\author{}
\date{}

\begin{document}

\maketitle

\section{Giới thiệu}

Trong nghiên cứu này, chúng tôi tập trung vào bài toán phân bổ nguồn lực cho nhiều dự án diễn ra đồng thời. Một trong những yêu cầu quan trọng của bài toán là \textbf{nhân viên không được làm việc trên nhiều dự án khác nhau trong cùng một ngày}, nhằm đảm bảo sự tập trung và hiệu quả. Bài toán này mở rộng từ vấn đề phân bổ nguồn lực cho một dự án duy nhất bằng cách tích hợp thêm các ràng buộc phức tạp liên quan đến quản lý đa dự án.

Mục tiêu chính là tối ưu hóa việc sử dụng tài sản, nhân lực và thời gian, đồng thời thỏa mãn các ràng buộc về \textbf{thời gian hoàn thành}, \textbf{KPI (Key Performance Indicators)} của từng dự án, và \textbf{chi phí vận hành}. Các đóng góp chính của nghiên cứu bao gồm:
\begin{itemize}
    \item Điều chỉnh cấu trúc dữ liệu để hỗ trợ quản lý đồng thời nhiều dự án.
    \item Phát triển thuật toán lập lịch tổng hợp cho tất cả các công việc từ nhiều dự án.
    \item Mở rộng thuật toán DLHS (Dựa trên Harmony Search) để gán nhân lực với ràng buộc không làm việc đa dự án trong cùng ngày.
    \item Đề xuất cơ chế xử lý xung đột thời gian nhằm đảm bảo tính khả thi của lịch trình.
\end{itemize}

\section{Cấu trúc dữ liệu}

Để quản lý hiệu quả nhiều dự án, chúng tôi đã mở rộng cấu trúc dữ liệu như sau:
\begin{itemize}
    \item \textbf{Công việc}: Mỗi công việc $T_i$ được bổ sung trường \texttt{project\_Id} để chỉ định dự án mà nó thuộc về. Ví dụ:
    \begin{align*}
    \{
    T_1: \{ & \text{project\_Id}: P1, \\
            & \text{duration}: 2, \\
            & \text{predecessors}: [] \}, \\
    T_2: \{ & \text{project\_Id}: P2, \\
            & \text{duration}: 3, \\
            & \text{predecessors}: [T_1] \} 
    \}
    \end{align*}
    \item \textbf{Nhân viên và tài sản}: Trạng thái của nhân viên ($E_i$) và tài sản ($A_i$) được theo dõi trên tất cả các dự án, bao gồm thời gian làm việc và dự án liên quan. Ví dụ:
    \[
    \{
    E_1: \{
        \text{time}: [t1, t2], 
        \text{project\_Id}: P1
    \}, 
    A_1: \{
        \text{time}: [t3, t4], 
        \text{project\_Id}: P2
    \} 
    \}
    \]
\end{itemize}

Cấu trúc này cho phép theo dõi và quản lý nguồn lực một cách linh hoạt giữa các dự án.

\section{Sắp xếp và lập lịch}

Quá trình sắp xếp và lập lịch là nền tảng để tổ chức các công việc trong nhiều dự án, đảm bảo chúng được thực hiện theo thứ tự hợp lý và tối ưu hóa việc sử dụng nguồn lực. Phần này bao gồm hai giai đoạn chính: **sắp xếp topo cho từng dự án** và **lập lịch tổng hợp cho tất cả các công việc**. Dưới đây là chi tiết của từng giai đoạn, kèm theo công thức và mã giả.

\subsection{Sắp xếp topo cho từng dự án}
Trong mỗi dự án, các công việc có thể có quan hệ phụ thuộc, nghĩa là một số công việc phải hoàn thành trước khi công việc khác có thể bắt đầu. Để giải quyết vấn đề này, chúng ta sử dụng \textbf{thuật toán sắp xếp topo (Topological Sort)} cho từng dự án dựa trên trường \texttt{project\_Id}.
\subsubsection{Lý do sử dụng sắp xếp topo}

Sắp xếp topo được chọn vì:
- **Đảm bảo thứ tự thực hiện hợp lệ**: Thuật toán tạo ra một thứ tự tuyến tính cho các công việc, trong đó mỗi công việc $T_i$ chỉ bắt đầu sau khi tất cả các công việc tiền nhiệm $T_i^p$ đã hoàn thành.
- **Phát hiện vòng lặp**: Nếu có vòng lặp trong quan hệ phụ thuộc (ví dụ: \( T_1 \to T_2 \to T_3 \to T_1 \)), thuật toán sẽ phát hiện và báo lỗi.

\paragraph{Quy trình thực hiện}

Quy trình sắp xếp topo cho từng dự án bao gồm các bước sau:
1. **Nhóm công việc theo dự án**:
    - Phân loại tất cả công việc \( \{T_i\} \) theo \texttt{project\_Id}.
    - Ví dụ:
      \[
      \text{tasksByProject} = \{ \text{P1}: [T_1, T_3, T_5], \text{P2}: [T_2, T_4, T_6] \}
      \]
2. **Áp dụng thuật toán sắp xếp topo**:
    - Với mỗi dự án \( P_j \), thực hiện thuật toán topo:
      - **Bước 1**: Tính *indegree* (số công việc tiền nhiệm) cho mỗi công việc.
      - **Bước 2**: Khởi tạo một hàng đợi với các công việc có *indegree* = 0.
      - **Bước 3**: Lặp lại cho đến khi hàng đợi rỗng:
         - Lấy công việc từ hàng đợi, thêm vào danh sách thứ tự.
         - Giảm *indegree* của các công việc phụ thuộc vào nó.
         - Nếu *indegree* của công việc phụ thuộc giảm xuống 0, thêm vào hàng đợi.
      - **Bước 4**: Nếu không sắp xếp được tất cả công việc, báo lỗi do có vòng lặp.
3. **Kết quả**:
    - Một danh sách thứ tự công việc cho mỗi dự án, ví dụ:
      \[
      \text{sortedTasks} = \{ P1: [T_1, T_3, T_5], P2: [T_2, T_4, T_6] \}
      \]

\subsubsection*{Mã giả cho sắp xếp topo}

Dưới đây là mã giả cho hàm `TopologicalSort` áp dụng cho một dự án:

\begin{algorithm}
\begin{algorithmic}[1]
\Function{TopologicalSort}{tasks}
     \State indegree $\gets$ tính indegree cho mỗi $T_i$ trong tasks
     \State queue $\gets$ các $T_i$ có indegree = 0
     \State result $\gets$ danh sách rỗng
     \While{queue không rỗng}
          \State $T_i$ $\gets$ lấy công việc từ queue
          \State thêm $T_i$ vào result
          \For{each $T_j$ phụ thuộc vào $T_i$}
                \State indegree[$T_j$] $\gets$ indegree[$T_j$] - 1
                \If{indegree[$T_j$] = 0}
                     \State thêm $T_j$ vào queue
                \EndIf
          \EndFor
     \EndWhile
     \If{số công việc trong result < số công việc trong tasks}
          \State \Return "Lỗi: Có vòng lặp trong dự án"
     \EndIf
     \State \Return result
\EndFunction
\end{algorithmic}
\end{algorithm}

\subsection{Lập lịch tổng hợp}

Sau khi có thứ tự topo cho từng dự án, chúng ta gộp tất cả công việc từ các dự án và sử dụng **phương pháp đường găng (Critical Path Method - CPM)** để lập lịch tổng hợp. CPM giúp xác định thời gian thực hiện tối ưu và quản lý các ràng buộc giữa các dự án.

\subsubsection*{Lý do sử dụng CPM}

CPM được chọn vì:
- **Xác định thời gian khả thi**: Tính toán thời gian bắt đầu sớm nhất (ES), kết thúc sớm nhất (EF), bắt đầu muộn nhất (LS), và kết thúc muộn nhất (LF) cho mỗi công việc.
- **Tối ưu hóa lịch trình**: Xác định đường găng và các công việc có thể thực hiện song song.

\subsubsection*{Quy trình thực hiện}

Quy trình lập lịch tổng hợp bao gồm các bước sau:
1. **Gộp tất cả công việc**:
    - Hợp nhất các danh sách công việc đã sắp xếp topo từ tất cả dự án thành một danh sách lớn.
    - Ví dụ:
      \[
      \text{allTasks} = [T_1, T_2, T_3, T_4, T_5, T_6]
      \]
2. **Tính toán ES và EF (Forward Pass)**:
    - Với mỗi công việc \( T_i \):
      - **Thời gian bắt đầu sớm nhất (ES)**:
         \[
         ES(T_i) = 
         \begin{cases} 
         \max_{T_j \in T_i^p} EF(T_j) & \text{nếu } T_i^p \neq \emptyset \\
         P_j^s & \text{nếu } T_i^p = \emptyset
         \end{cases}
         \]
         Trong đó \( P_j^s \) là thời gian bắt đầu của dự án \( P_j \) chứa \( T_i \), \( T_i^p \) là tập hợp các công việc tiền nhiệm của \( T_i \).
      - **Thời gian kết thúc sớm nhất (EF)**:
         \[
         EF(T_i) = ES(T_i) + T_i^d
         \]
         Trong đó \( T_i^d \) là thời gian thực hiện của công việc \( T_i \).
3. **Tính toán LF và LS (Backward Pass)**:
    - Với mỗi công việc \( T_i \):
      - **Thời gian kết thúc muộn nhất (LF)**:
         \[
         LF(T_i) = 
         \begin{cases} 
         \min_{T_j \in T_i^{\text{successor}}} LS(T_j) & \text{nếu } T_i^{\text{successor}} \neq \emptyset \\
         P_j^e & \text{nếu } T_i^{\text{successor}} = \emptyset
         \end{cases}
         \]
         Trong đó \( T_i^{\text{successor}} \) là các công việc tiếp theo \( T_i \), \( P_j^e \) là thời gian kết thúc dự kiến của dự án \( P_j \).
      - **Thời gian bắt đầu muộn nhất (LS)**:
         \[
         LS(T_i) = LF(T_i) - T_i^d
         \]
4. **Xác định đường găng và độ trễ cho phép**:
    - **Đường găng**: Các công việc có \( ES(T_i) = LS(T_i) \) và \( EF(T_i) = LF(T_i) \).
    - **Độ trễ cho phép (slack)**:
      \[
      \text{slack}(T_i) = LS(T_i) - ES(T_i)
      \]

\subsubsection*{Mã giả cho CPM}

Dưới đây là mã giả cho hàm `ApplyCPM` để lập lịch tổng hợp:

\begin{algorithm}
\begin{algorithmic}[1]
\Function{ApplyCPM}{allTasks}
     \State \Comment{Forward Pass}
     \For{each $T_i$ in allTasks (theo thứ tự topo)}
          \If{$T_i^p$ rỗng}
                \State $ES(T_i) \gets P_j^s$ \Comment{thời gian bắt đầu của dự án $P_j$ chứa $T_i$}
          \Else
                \State $ES(T_i) \gets \max_{T_j \in T_i^p} EF(T_j)$
          \EndIf
          \State $EF(T_i) \gets ES(T_i) + T_i^d$
     \EndFor
     \State \Comment{Backward Pass}
     \For{each $T_i$ in allTasks (ngược thứ tự topo)}
          \If{$T_i^{\text{successor}}$ rỗng}
                \State $LF(T_i) \gets P_j^e$ \Comment{thời gian kết thúc của dự án $P_j$ chứa $T_i$}
          \Else
                \State $LF(T_i) \gets \min_{T_j \in T_i^{\text{successor}}} LS(T_j)$
          \EndIf
          \State $LS(T_i) \gets LF(T_i) - T_i^d$
     \EndFor
     \State \Comment{Xác định đường găng và độ trễ}
     \For{each $T_i$ in allTasks}
          \If{$ES(T_i) = LS(T_i)$ và $EF(T_i) = LF(T_i)$}
                \State đánh dấu $T_i$ là công việc trên đường găng
          \EndIf
          \State $\text{slack}(T_i) \gets LS(T_i) - ES(T_i)$
     \EndFor
\EndFunction
\end{algorithmic}
\end{algorithm}

\subsection{Kết quả của quá trình sắp xếp và lập lịch}

Sau khi hoàn tất, chúng ta thu được:
- **Thứ tự thực hiện công việc**: Đảm bảo các công việc trong từng dự án được thực hiện theo thứ tự hợp lệ.
- **Khung thời gian khả thi**: Các giá trị ES, EF, LS, LF giúp xác định khoảng thời gian linh hoạt cho mỗi công việc, đồng thời xác định các công việc quan trọng trên đường găng.

Những kết quả này là cơ sở cho việc gán tài nguyên và nhân lực trong các bước tiếp theo, đảm bảo lịch trình được xây dựng một cách logic và tối ưu.

\subsection{Sắp xếp topo cho từng dự án}
Đầu tiên, chúng tôi thực hiện \textbf{sắp xếp topo} (topological sort) cho từng dự án dựa trên trường \texttt{project\_Id}. Điều này đảm bảo rằng các công việc tiền nhiệm ($T_i^p$) phải được hoàn thành trước khi bắt đầu các công việc phụ thuộc. Kết quả là danh sách thứ tự công việc cho mỗi dự án:
\[
\{ P1: [T_1, T_3, T_5], P2: [T_2, T_4, T_6] \}
\]

\subsection{Lập lịch tổng hợp}
Sau đó, tất cả công việc từ các dự án được gộp thành một tập hợp lớn. Chúng tôi áp dụng \textbf{phương pháp CPM (Critical Path Method)} để tính toán các thông số thời gian quan trọng:
\begin{itemize}
    \item \textbf{ES (Earliest Start)}: Thời gian bắt đầu sớm nhất.
    \item \textbf{EF (Earliest Finish)}: Thời gian kết thúc sớm nhất.
    \item \textbf{LS (Latest Start)}: Thời gian bắt đầu muộn nhất.
    \item \textbf{LF (Latest Finish)}: Thời gian kết thúc muộn nhất.
\end{itemize}

Phương pháp này giúp xác định khung thời gian khả thi cho từng công việc, đồng thời xem xét các ràng buộc giữa các dự án.

\section{Gán tài sản và khung thời gian}

\subsection{Gán tài sản}
Chúng tôi duy trì một \textbf{lịch sử sử dụng tài sản} trên tất cả các dự án. Khi gán tài sản cho công việc $T_i$:
\begin{enumerate}
    \item Kiểm tra thời gian khả dụng của tài sản dựa trên lịch sử.
    \item Chọn tài sản phù hợp nhất (ít xung đột nhất).
    \item Cập nhật trạng thái tài sản sau khi gán.
\end{enumerate}

Ví dụ: Nếu $A_1$ trống từ $t_5$ đến $t_7$, nó sẽ được gán cho $T_3$ nếu khung thời gian phù hợp.

\subsection{Gán khung thời gian}
Thời gian bắt đầu ($T_i^s$) và kết thúc ($T_i^e$) của công việc $T_i$ được tính toán dựa trên:
\begin{itemize}
    \item Thời gian khả dụng của tài sản đã gán.
    \item Thời gian hoàn thành của các công việc tiền nhiệm trong cùng dự án.
\end{itemize}

Điều này đảm bảo không có vi phạm về thứ tự thực hiện hoặc sự chồng lấn tài nguyên.

\section{Gán nhân lực}

\subsection{Phân chia KPI cho nhiều dự án}
KPI mục tiêu được phân bổ cho từng dự án, ví dụ:
\[
\{ P1: \text{KPI}_1 = 80\%, P2: \text{KPI}_2 = 90\% \}
\]
Sau đó, KPI của từng dự án được phân bổ cho nhân viên dựa trên năng lực của họ, với ràng buộc: \textbf{mỗi nhân viên chỉ làm việc cho một dự án duy nhất trong một ngày}.

\subsection{Gán nhân lực bằng thuật toán DLHS}
Chúng tôi sử dụng thuật toán \textbf{DLHS} (Dựa trên Harmony Search) để gán nhân lực:
\begin{itemize}
    \item Một \textbf{vector hài hòa} đại diện cho phương án gán nhân lực cho tất cả công việc của tất cả dự án.
    \item Trong quá trình tạo phương án cải tiến, kiểm tra để đảm bảo không nhân viên nào làm việc trên nhiều dự án cùng ngày.
\end{itemize}

\section{Xử lý xung đột thời gian}

Sau khi gán nhân lực, chúng tôi kiểm tra lịch trình để phát hiện xung đột:
\begin{itemize}
    \item Nếu nhân viên $E_i$ được gán cho $T_1$ (dự án $P1$) và $T_2$ (dự án $P2$) trong cùng ngày, xung đột xảy ra.
    \item \textbf{Giải pháp}: Dời công việc của dự án ít ưu tiên hơn sang ngày tiếp theo. Cơ chế ưu tiên dựa trên:
    \begin{enumerate}
        \item Độ trễ cho phép (slack time) của công việc.
        \item Tầm quan trọng của dự án (được định nghĩa trước).
    \end{enumerate}
\end{itemize}

\section{Tối ưu hóa tổng thể}

Chúng tôi định nghĩa một \textbf{hàm mục tiêu tổng thể} để đánh giá chất lượng phân bổ nguồn lực:
\[
F = w_1 \cdot \text{Total Time} + w_2 \cdot \text{Total Cost} + w_3 \cdot \text{KPI Deviation} + w_4 \cdot \text{Conflicts}
\]
Trong đó $w_1, w_2, w_3, w_4$ là các trọng số. Thuật toán DLHS được điều chỉnh để tối ưu hóa đồng thời cho nhiều dự án, sử dụng cơ chế trao đổi thông tin giữa các \textbf{sub-HM} (sub-Harmony Memories) nhằm cải thiện hiệu quả tìm kiếm.

\section{Thuật toán điều chỉnh}

Dưới đây là mã giả của thuật toán DLHS điều chỉnh:

\begin{algorithm}
\caption{DLHSMultiProjects}
\begin{algorithmic}[1]
\State \textbf{Input:} $T_i, E_i, A_i, KPI_{target}, DLHS_{Args}$
\State \textbf{Output:} Assignment of resources
\State \textbf{Step 1:} Sắp xếp topo cho từng dự án
\For{each $P_j$ in projects}
    \State $sortedTasks[P_j] \gets$ TopologicalSort($tasksByProject[P_j]$)
\EndFor
\State \textbf{Step 2:} Lập lịch tổng hợp
\State $allTasks \gets$ Flatten($sortedTasks$)
\State Apply CPM on $allTasks$ to get ES, EF, LS, LF
\State \textbf{Step 3 \& 4:} Gán tài sản và thời gian
\For{each $T_i$ in $allTasks$}
    \State Assign assets and calculate $T_i^s, T_i^e$
\EndFor
\State \textbf{Step 5 \& 6:} Gán nhân lực với DLHS
\State Initialize harmony memory (HM)
\While{not converged}
    \State Generate new solutions and update HM
\EndWhile
\State \textbf{Step 7:} Xử lý xung đột
\For{each $E_i$ in $E_i$}
    \State Resolve conflicts across projects
\EndFor
\State \textbf{Return:} Final assignment
\end{algorithmic}
\end{algorithm}

\section{Kết luận}

Nghiên cứu này đã trình bày một phương pháp toàn diện để phân bổ nguồn lực cho nhiều dự án, với ràng buộc đặc biệt rằng nhân viên không làm việc trên nhiều dự án cùng ngày. Bằng cách điều chỉnh cấu trúc dữ liệu, thuật toán lập lịch, và thuật toán DLHS, chúng tôi đã phát triển một giải pháp hiệu quả, khả thi và tối ưu. Kết quả cho thấy phương pháp có thể xử lý các ràng buộc phức tạp và cung cấp lịch trình chất lượng cao. Trong tương lai, chúng tôi dự kiến khám phá việc tích hợp các yếu tố động như thay đổi tiến độ hoặc sự sẵn có của nguồn lực.

\end{document}
