% - Nên có 1 module, giải thuật chạy local, độc lập riêng với đầu vào là các File, đầy ra là các File hoặc lịch cơ bản
% - Ghép nối ( tích hợp với dxclan)
Vậy sẽ
% - Hoàn thiện thuật toán:
% - Đầu vào :
% - 1 tập dự án
% - 1 tập nhân viên - năng lực liên quan
% - 1 tập công việc trong dự án kèm theo yêu cầu
% - Đầu ra:
% - Lịch của toàn bộ công ty, nhân sự
% - Lịch cho từng dự án
% - Lịch cho từng nhân viên.
% - Nhân viên cũng cần theo dõi xem tham gia dự án nào
% - Tích hợp vào dxclan:

% Công việc to -> báo cáo n lần ->kpi là hợp của các công việc con
% Công việc to -> công việc con -> định nghĩa ràng buộc theo kiều gì
% Cả hai cái trên đều ok, mình hỗ trợ cái nào hay cả 2

% Đầu vào , hướng dẫn đánh giá KPI ,

% Đầu ra: ngoài lịch ra còn có File dự báo kpi của từng người


% Vấn đề quan trọng :
% Đặc tả đầu vào đầu ra cho nó đúng
% - Hiển thị lịch cho nhân viên tham gia vào nhiều dự án
% - Kế hoạch phân bổ kpi cho nhân viên trong nhiều dự án. Với mỗi kpi thì phải thực hiện công việc gì của dự án nào để được kpi đó
% - Sau đó : xem kpi có cân bằng không,….( để sau).

% Đánh giá Kpi phân bổ cho những người có Profile khác nhau nó có bị lệch không?
% Ẩn bớt
\documentclass{article}
\usepackage[utf8]{vietnam}
\usepackage{graphicx}
\usepackage{xcolor}
\usepackage{array} % Added to fix tabular issues
\usepackage{booktabs}%
\usepackage{algorithm2e}%
\usepackage{algpseudocode}%
\usepackage{multirow}
\usepackage{natbib,stfloats}
\usepackage{mathrsfs}

\title{Báo cáo tuần 5}
\author{}
\date{}

\begin{document}

\maketitle

\section{Giới thiệu chung}

Trong nghiên cứu này, chúng tôi tập trung vào bài toán phân bổ nguồn lực cho nhiều dự án diễn ra đồng thời. Một trong những yêu cầu quan trọng của bài toán là \textbf{nhân viên không được làm việc trên nhiều dự án khác nhau trong cùng một ngày}, nhằm đảm bảo sự tập trung và hiệu quả. Bài toán này mở rộng từ vấn đề phân bổ nguồn lực cho một dự án duy nhất bằng cách tích hợp thêm các ràng buộc phức tạp liên quan đến quản lý đa dự án.

Mục tiêu chính là tối ưu hóa việc sử dụng tài sản, nhân lực và thời gian, đồng thời thỏa mãn các ràng buộc về \textbf{thời gian hoàn thành}, \textbf{KPI (Key Performance Indicators)} của từng dự án, và \textbf{chi phí vận hành}. Các đóng góp chính của nghiên cứu bao gồm:
\begin{itemize}
    \item Điều chỉnh cấu trúc dữ liệu để hỗ trợ quản lý đồng thời nhiều dự án.
    \item Phát triển thuật toán lập lịch tổng hợp cho tất cả các công việc từ nhiều dự án.
    \item Mở rộng thuật toán DLHS (Dựa trên Harmony Search) để gán nhân lực với ràng buộc không làm việc đa dự án trong cùng ngày.
    \item Đề xuất cơ chế xử lý xung đột thời gian nhằm đảm bảo tính khả thi của lịch trình.
\end{itemize}

\section{Cấu trúc dữ liệu}
\subsection{Inputs (Đầu vào)}
    Đầu vào của bài toán bao gồm các thông tin sau: 
        \begin{table} [htbp]
            \caption{Đầu vào thông tin chung về danh sách dự án}
            \centering
            \begin{tabular}{p{0.22\linewidth} p{0.3\linewidth} p{0.53\linewidth}} 
                \toprule
                \multicolumn{3}{c}{\textbf{Dự án} \textit{$P_{i}\epsilon P$}}\\
                \textbf{ID} & $P_i^{id}$ & 1\\ 
                \textbf{Tên dự án} & $P_i^n$ & Dự án phát triển PM prj-prp\\ 
                \textbf{Mô tả} & $P_i^d$ & Phần mềm hỗ trợ phân bổ nguồn lực dự án\\ 
                \textbf{Thời gian bắt đầu dự kiến} & $P_i^s$ & 8:00 01/08/2024\\ 
                \textbf{Thời gian kết thúc dự kiến} & $P_i^e$ & 8:00 01/11/2024\\ 
                \textbf{Danh sách chỉ tiêu KPI} & $P_i^{kpi} = \lbrace K_i \rbrace$ & \\
                \textbf{Danh sách công việc} & \textcolor{red}{$T_{assign} = \{T_i\}$} & \\
             \textbf{Danh sách nhân viên tham gia} & \textcolor{red}{$E_{assign} = \{E_i\}$} & \\
            
                \bottomrule             
            \end{tabular}
            \label{tab:table_input_thongtinchung}
        \end{table}
    Thông tin về chỉ tiêu KPI của dự án được mô tả trong Bảng \ref{tab:table_input_thongtinchitieukpi}
        \begin{table} [htbp]
            \caption{Đầu vào thông tin chung về chỉ tiêu KPI}
            \centering
            \begin{tabular}{p{0.22\linewidth} p{0.3\linewidth} p{0.53\linewidth}} 
                \toprule
                \multicolumn{3}{c}{\textbf{Chỉ tiêu KPI} \textit{$K_{i}\epsilon K$}}\\
                \textbf{ID} & $K_i^{id}$ & 1\\ 
                \textbf{Tên KPI} & $K_i^n$ & Phân tích, phát triển chức năng, mô-đun\\
                \textbf{Tiêu chí đánh giá} & $K_i^c$ & Số lượng mô-đun hoàn thành đúng tiến độ\\
                \textbf{Đơn vị giá trị } & $K_i^u$ & mô-đun\\
                \textbf{Tổng giá trị giao} & $K_i^total$ & 100\%\\ 
                \textbf{Giá trị cần đạt} & $K_i^require$ & 80\%\\
                \textbf{Giá trị KPI mục tiêu} & $K_i^tv$ & 0.8\\
                \bottomrule             
            \end{tabular}
            \label{tab:table_input_thongtinchitieukpi}
        \end{table}
\begin{table} [htbp]
    \caption{Đầu vào danh sách công việc trong dự án}
    {\begin{tabular}{>{\raggedright\arraybackslash}p{0.26\linewidth}>{\raggedright\arraybackslash}p{0.17\linewidth}>{\raggedright\arraybackslash}p{0.5\linewidth}} \toprule
         \multicolumn{3}{c}{\textbf{Công việc} \textit{P}}\\ %\midrule 
         Tên công việc& $T_i^n$ &Dựng trang thống kê dự án\\ 
         Mã công việc&$T_i^c$ &A4\\   
         Thẻ công việc (tags) &$T_i^{tags}$ &frontend\\   
         Công việc tiền nhiệm& $T_i^{p}=\{t_i\} \subseteq \{T_i^c\}$&A2, A3\\   
         Ước tính thời gian thực hiện& $T_i^d$ &3 ngày\\   
         Yêu cầu năng lực nhân viên& $re_i$ & (1, Lập trình viên, (Back-end, thành thạo), bắt buộc)\\   
         Yêu cầu năng lực tài sản& $ra_i $ & (2, Laptop, {(RAM, 32GB), (Disk, 100GB)}, tùy chọn)\\   
         KPI tương ứng công việc& $T_i^{kpi}$ &\\   \midrule  
         \multicolumn{3}{c}{\textbf{Yêu cầu về năng lực nhân viên} $re_{ij}= (q, \{(re_{ij}^{qa}, re_{ij}^{qv})\}, o) \in re_i$}\\ %\midrule  
         Số lượng&$ra_i^{n}$ &1\\   
         Vai trò &$ra_i^t$& Lập trình viên\\   
         Năng lực yêu cầu &$re_{ij}^{qa}$ &Backend\\   
         Giá trị năng lực yêu cầu&$re_{ij}^{qv}$ &Thành thạo\\   
         Loại yêu cầu&$ra_i^{rt}$ &Bắt buộc\\ \midrule
         \multicolumn{3}{c}{\textbf{Yêu cầu về năng lực tài sản} $ra_{ij} = (q, \{(ra_{ij}^{qa}, ra_{ij}^{qv})\}, o) \in ra_i$}\\   %\midrule
         Số lượng&$ra_{ij}^{n}$ &1\\   
         Vai trò&$ra_{ij}^t$ &Laptop\\   
         Năng lực yêu cầu &$ra_{ij}^{qa}$ &RAM\\   
         Giá trị năng lực yêu cầu&$ra_{ij}^{qv}$ &32GB\\   
         Loại yêu cầu&$ra_{ij}^{rt}$ &Bắt buộc\\   \midrule         
         \multicolumn{3}{c}{\textbf{KPI tương ứng với task} $T_i^{kpi}=(k_i^{id}, k_i^{w})$}\\ %  \midrule
         ID của KPI&$k_i^{id} \in \{k_i\}$& 1 \\  
         Trọng số &$k_i^{w}$ &0.125\\ 
         \bottomrule             
    \end{tabular}}
    \label{tab:table_input_task_list}
\end{table}


    \begin{table} [htbp]
        \caption{Đầu vào danh sách nhân viên tham gia dự án}
        {\begin{tabular}{>{\raggedright\arraybackslash}p{0.32\linewidth}>{\raggedright\arraybackslash}p{0.12\linewidth}>{\raggedright\arraybackslash}p{0.57\linewidth}} \toprule
             \multicolumn{3}{c}{\textbf{Tài nguyên} $E_i \in E$}\\ %\midrule
            %\textbf{Thuộc tính}& \textbf{Ký hiệu}&\textbf{Ví dụ}\\ \midrule 
             ID& $E_i^{id}$&H0001\\   
            Tên tài nguyên& $E_i^n$ &Nguyễn Văn M\\  
            Mã tài nguyên&$E_i^c$ &SW20210901\\   
            Loại tài nguyên& $E_i^t$& \{Nhân viên, Chuyên gia ATTT, ...\}\\  
            Chi phí vận hành / lương & $E_i^s$&10,000,000 VNĐ\\  
            Tập năng lực & $E_i^q=\{E_{ij}^q\}$&\\  \midrule
            \multicolumn{3}{c}{\textbf{Năng lực} $E_{ij}^q = (E_{ij}^{qa}, E_{ij}^{qv}) \in E_i^q$}\\ %\hline 
            %\textbf{Thuộc tính}& \textbf{Ký hiệu}&\textbf{Ví dụ}\\ \hline 
            Tên năng lực& $E_{ij}^{qa}$&Backend\\  
            Giá trị năng lực& $E_{ij}^{qv}$&Chuyên nghiệp\\ 
            Tập kinh nghiệm & $E_i^q=\{E_{ij}^q\}$&\\  \midrule
            \multicolumn{3}{c}{\textbf{Kinh nghiệm} $E_{ij}^q = (E_{ij}^{qa}, E_{ij}^{qv}) \in E_i^q$}\\ %\hline 
           % \textbf{Thuộc tính}& \textbf{Ký hiệu}&\textbf{Ví dụ}\\ \hline 
            Tên kinh nghiệm (tag) & $E_{ij}^{qa}$&Lập trình C\\  
            Hiệu suất thực hiện trong quá khứ & $E_{ij}^{qv}$&0.8\\ 
             \bottomrule             
        \end{tabular}}
        %\tabnote{\textsuperscript{a}This footnote shows how to include footnotes to a table if required.}
        \label{tab:table_resource}
    \end{table}



\end{document}
